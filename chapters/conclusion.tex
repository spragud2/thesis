The non-coding portion of our genome represents a significant fraction of the mammalian transcriptome \cite{Rinn2012GenomeRNAs}. Small non-coding RNAs are relatively well understood \cite{Rinn2012GenomeRNAs}, however they make up only a fraction of annotated non-coding RNA transcripts. Despite representing a large fraction of known transcripts in the mammalian genome, we have solid understanding of only a handful of lncRNAs. \emph{XIST} was discovered in 1992 \cite{Brown10TheNucleus.}, and yet we are still learning how exactly Xist functions within the cell \cite{Schertzer2019LncRNA-InducedDNA,DaRocha2017NovelConformation}. 

A significant frustration is the unclear sequence to function relationship in lncRNAs. This relationship has long been understood in protein coding genes \cite{Whisstock2003PredictionStructure} and has been facilited by numerous bioinformatics tools that can rapidly predict relationships between protein coding genes \cite{Altschul1990BasicTool,Smith1981IdentificationSubsequences,Wheeler2013Nhmmer:HMMs,Yi2013Co-phylog:Organisms,Qi2004WholeApproach}. Relatively speaking though very little is known about lncRNAs. It is known that the tandem repeats of \emph{XIST} encode a significant amount of its function \cite{Nesterova2001CharacterizationSequence,Wang2017TargetingGuanines,Hoki2009AMouse,Zhao2008PolycombChromosome,Pintacuda2017HnRNPKSilencing}, despite much of the sequence of \emph{XIST} being relatively poorly conserved \cite{Nesterova2001CharacterizationSequence}. This has led to the notion that lncRNAs may derive a significant amount of their function from small sub-sequences \cite{Brockdorff2018LocalNcRNA, Sprague2019NonlinearDomains}. 

Our lab developed a $k$-mer based algorithm for quantifying sequence similarity between two lncRNAs \cite{Kirk2018FunctionalContent}. This revealed large communities of lncRNAs that are related based on their $k$-mer content, and these communities were predictive of cellular function \cite{Kirk2018FunctionalContent}. However, the functional analog to \emph{XIST} in marsupials, \emph{Rsx} was anti-correlated with \emph{XIST} in this analysis. This observation led to the implementation of a modular approach to lncRNA sequence analysis -- \emph{i.e.}, breaking the sequences up into known or hypothesized functional domains and comparing only those regions. It was only after this that the degree of similarity between \emph{XIST} and \emph{Rsx} could be realized \cite{Sprague2019NonlinearDomains}. We then developed a statistical model using an HMM framework that was capable of identifying functional domains within lncRNAs that lack any sort of obvious tandem repeat domains as in \emph{XIST} and \emph{Rsx}. 

We now have a set of tools that can answer how and where the function is encoded within lncRNAs. There remain significant ways that they can be improved and areas in which further understanding is needed within lncRNAs. A primary shortcoming of SEEKR is that we do not consider secondary structure of the RNA when modeling its potential function. RNA molecules are able to form a wide variety of secondary structures in a sequence dependent manner that are known to be functional within the cell \cite{Siegfried2014RNASHAPE-MaP,Wan2011UnderstandingStructure}. Despite this, we are able to capture extensive RBP binding events in \emph{XIST} and \emph{KCNQ1OT1} with high precision and without considering structure at all. While structure is certainly important and has been shown to be conserved in lncRNAs \cite{Johnsson2014EvolutionaryFunction}, these results imply that overly focusing on secondary structure as the primary mechanism of lncRNA function may not tell the whole story.

Another limitation of \emph{hmmSEEKR} is the assumption of a 2-state HMM and the requirement for the \emph{a priori} definition of a query. Extending the logic behind what the \emph{XIST} queries represent, it is possible to generate a query-of-interest through a Bayesian sampling algorithm \cite{McLachlan2019FiniteModels}. One way to view the sequences of the functional domains of \emph{XIST} is as a mixture model of $k$-mers that ultimately correspond to subset of differing proteins that recognize that sequence region. Given a set of weights from which to sample RBP PWMs, an artificial query can be generated by sampling $k$-mers from the PWMs with frequencies determined by the weights given to each protein. A further extension of the HMM model removes the assumption on a 2-state system entirely by utilizing a hierarchical Dirichlet process HMM (HDP-HMM) \cite{Johnson2013BayesianModels}. Put briefly, the HDP-HMM models the number of hidden states within a sequence without any prior information on what the hidden states are. This allows the model to determine how many functional domains may be within a sequence, what the $k$-mer content of those functional domains is, and where these domains are within a sequence. 

lncRNAs represent a significant challenge in molecular biology -- both computationally and experimentally. However, they represent approximately a third of annotated transcripts in the human genome, so it is crucial that models of their function are developed. Given the complex nature of lncRNAs, and their roles as hubs within the cell, I expect that future developments in the field will extend beyond simply modeling the sequence of a lncRNA, as well as their secondary and tertiary structure. Rather, integrative models, that model the complex interactions between RNA, protein, and DNA, will push the field further. With proteins and mRNAs, this was unnecessary to prediction the function of a transcript -- because the protein was effectively the molecule performing a single function. With lncRNAs, again, their role is much more subtle and a piece in a larger puzzle. I also believe that as more knowledge is gained about lncRNA function, more advanced models can be proposed that incorporate more prior information. Perhaps there are sequence features that, due to lack of training data knowledge, are simply undetectable for the moment. Finally, further developments in understanding conservation of non-coding regions of the genome, and development of models that consider conservation of sequence features other than linear alignment, will be crucial to understanding the nature of lncRNAs in the cell. 

\begin{singlespace}
\printbibliography[heading=bibintoc,title={References}]
\end{singlespace}