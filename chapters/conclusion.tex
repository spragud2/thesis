The non-coding portion of our genome represents a significant portion of the mammalian transcriptome \cite{Rinn2012GenomeRNAs}. Small non-coding RNAs are relatively well understood \cite{Rinn2012GenomeRNAs}, however they make up only a fraction of annotated non-coding RNA transcripts. Despite representing a large fraction of known transcripts in the mammalian genome, we have solid understanding of only a handful of lncRNAs. \emph{XIST} was discovered in 1992 \cite{Brown10TheNucleus.}, and yet we are still learning how exactly Xist functions within the cell \cite{Schertzer2019LncRNA-InducedDNA,DaRocha2017NovelConformation}. 

A significant frustration is the unclear sequence to function relationship in lncRNAs. This relationship has long been understood in protein coding genes \cite{Whisstock2003PredictionStructure} and has been facilited by numerous bioinformatics tools that can rapidly predict relationships between protein coding genes \cite{Altschul1990BasicTool,Smith1981IdentificationSubsequences,Wheeler2013Nhmmer:HMMs,Yi2013Co-phylog:Organisms,Qi2004WholeApproach}. 

\begin{singlespace}
\printbibliography[heading=bibintoc,title={References}]
\end{singlespace}