\subsubsection{High throughput loss of function screens}
Perhaps the holy grail of lncRNA functional annotation is a reliable high throughput screening methodology to ascribe function to lncRNA transcripts. RNA interference studies have found that lincRNA knockdowns can have profound effects throughout the transcriptome, especially in the maintenance of stem of cells. RNAi screens suffer from several limitations, however, including variance in silencing and off-target effects. Within the realm of cancer biology, RNAi screens have led to several well known examples of erroneously ascribed function, including \emph{MELK} and \emph{STK33}.

Traditional \emph{CRISPR} approaches are not viable as lncRNAs lack reading frames, however over- and under-expression extensions such as \emph{CRISPRa} and \emph{CRISPRi} have identified functional lncRNAs \cite{Cai2020AHomeostasis}. Despite these successes, the correlative nature of these screens are just the first steps in elucidating the mechanisms by which a lncRNA function. 

Of the small number of lncRNAs that have well characterized roles within the cell, many still have their exact mechanism of action being actively studied. \emph{XIST} is one of the most well known lncRNAs and was discovered in the early 1990s, however there is still considerable controversy about how exactly it enacts XCI. \emph{XIST} serves as a reminder that these genes are very complicated and are often multi-faceted in their function, and so therefore there remains a massive amount of work left in the field to sift through the truly enormous number of annotated lncRNAs in the human genome.